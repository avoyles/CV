\begin{rSubsection}{University of California, Berkeley}{\DTMdisplaydate{2019}{6}{24} -- Present}{Assistant Research Engineer}{Berkeley, California}
\item Led fundamental studies of low-energy nuclear physics at the LBNL 88-Inch Cyclotron as a part of the Bay Area Nuclear Data Program,  and supervised M.S./Ph.D. students in these efforts.
\item Efforts include the measurement of charged-particle and neutron-induced reaction cross sections relevant to the production of radionuclides for medical applications, and the measurement of independent and cumulative fission yields using cyclical neutron activation analysis.
% \item Developed in-house capabilities for electrodeposition and  pressed-powder target fabrication.
\item Mentored M.S./Ph.D. students' research for cross section measurements and evaluations at LBNL, LANL, and BNL.
\item As  Isotope Production technical leader, responsible for developing the technical vision for these research objectives, 
% draft and revise technical papers and reports using sound scientific judgment, and prepare and make technical presentations at research and technical conferences on research progress. In addition, the candidate will 
and facilitating interactions with other research organizations  to promote collaboration and enhance the impact of research results, chiefly with LANL and BNL. 
\item Developed stable and radioactive target fabrication capabilities in support of these objectives.
\item Compiled all nuclear data produced in  experiments into the reaction database EXFOR. 
\item Led OJT, Integrated Safety Management, EHS, and safety controls for experimental activities in the Bay Area Nuclear Data Program. 
\end{rSubsection}\vspace{-1.5\baselineskip}
\begin{rSubsection}{}{\DTMdisplaydate{2018}{8}{15} -- \DTMdisplaydate{2019}{6}{23}}{Postdoctoral Scholar}{}
\item Responsible for overseeing the effort to determine novel production routes for \ce{^{225}Ac}, \ce{^{212}Pb}, \ce{^{68}Ge}, and \ce{^{236}Np}, through experiments at the LBNL 88-Inch Cyclotron as a part of the LBNL/UCB Nuclear Data Program.
\item Developed in-house capabilities for electrodeposition and  pressed-powder target fabrication.
\item Assisted other members of the group by supervising M.S./Ph.D. student efforts to determine isotope production routes through cross section measurements at LBNL, LANL, and BNL.
% \item Compiled all nuclear data produced in  experiments into the reaction database EXFOR. 
\end{rSubsection}\vspace{-1.5\baselineskip}
\begin{rSubsection}{}{\DTMdisplaydate{2014}{8}{21} -- \DTMdisplaydate{2018}{8}{10}}{Graduate Student Researcher / NRC Fellow}{}
\iffull{ \item  
Dissertation Title: \enquote{Nuclear Excitation Functions for the Production of Novel Medical Radionuclides} ---  
% Current Ph.D.~research focused on the 
measurement of cross-sections for neutron-induced and charged particle-induced reaction pathways for the production of emerging novel therapeutic and diagnostic medical radionuclides, with high specific activity.  } \else {\item  
Researched \enquote{Nuclear Excitation Functions for Production of Novel Medical Radionuclides} ---  
% Current Ph.D.~research focused on the 
measurement of cross-sections for neutron-induced and charged particle-induced reaction pathways for the production of emerging novel therapeutic and diagnostic medical radionuclides, with high specific activity.  } \fi 
\iffull{ \item Dissertation Advisor:  Dr. Lee A. Bernstein, University of California, Berkeley}  \else \fi
\item  Developed intense mono-energetic neutron source capabilities for production of novel therapeutic radionuclides.
\item  Research carried out at 
% the combination of 
the Lawrence Berkeley National Laboratory's 88-Inch Cyclotron and the Los Alamos National Laboratory's  Isotope Production Facility at LANSCE.
% \item  Other research includes: 
% \begin{list2}
% \item  Development of intense mono-energetic neutron source capabilities for production of novel therapeutic radionuclides.
% \item  Evaluated solid debris collection diagnostics in search of evidence of nuclear-plasma interactions, at Osaka University Institute for Laser Engineering, Osaka, Japan.
% \end{list2}
\end{rSubsection}

% {\bf University of California, Berkeley} \hfill{ Berkeley, California}
% 
% \vspace*{-.05in}
% {\em Graduate Student Researcher / NRC Fellow} \hfill {\bf August, 2014 -- Present}
% 
% \begin{list2}
% \item  
% Researched \enquote{Nuclear Excitation Functions for Production of Novel Medical Radionuclides} ---  
% % Current Ph.D.~research focused on the 
% measurement of cross-sections for neutron-induced and charged particle-induced reaction pathways for the production of emerging novel therapeutic and diagnostic medical radionuclides, with high specific activity.  
% \item  Developed intense mono-energetic neutron source capabilities for production of novel therapeutic radionuclides.
% \item  Research carried out at 
% % the combination of 
% the Lawrence Berkeley National Laboratory's 88-Inch Cyclotron and the Los Alamos National Laboratory's  Isotope Production Facility at LANSCE.
% % \item  Other research includes: 
% % \begin{list2}
% % \item  Development of intense mono-energetic neutron source capabilities for production of novel therapeutic radionuclides.
% % \item  Evaluation of solid debris collection diagnostics in search 
% % of evidence of nuclear-plasma interactions, at Osaka University Institute for Laser Engineering, Osaka, Japan.
% % \end{list2}
% \end{list2}

\begin{rSubsection}{University of Oslo}{{\DTMsetdatestyle{myDateRange}\DTMdisplaydate{2018}{2}{12}} -- \DTMdisplaydate{2018}{5}{1}}{Visiting Researcher, Department of Physics}{Oslo, Norway}
\item    Studied preparation of a chelate-conjugated biomolecule carrying a radionuclide, in the Nuclear and Energy Physics group. 
\item    Focus on the radiolanthanide \ce{^{161}Tb} and a peptidomimetic displaying dual-receptor targeting through the endothelial growth factor receptor and the HER2/neu antigen.
\end{rSubsection}
% {\bf University of Oslo} \hfill{ Oslo, Norway}
% 
% \vspace*{-.05in}
% {\em Visiting Researcher, Department of Physics} \hfill {\bf April -- May,  2018}
% 
% \begin{list2}
% \item    Studied preparation of a chelate-conjugated biomolecule carrying a radionuclide, in the Nuclear and Energy Physics group. 
% \item    Focus on the radiolanthanide \ce{^{161}Tb} and a peptidomimetic displaying dual-receptor targeting through the endothelial growth factor receptor and the HER2/neu antigen.
% \end{list2}



\begin{rSubsection}{Institute for Laser Engineering, Osaka University}{{\DTMsetdatestyle{myDateRange}\DTMdisplaydate{2015}{2}{13}} -- \DTMdisplaydate{2015}{3}{3}}{Visiting Researcher}{Osaka, Japan}
\item    Research and evaluation of solid debris collection diagnostics in search of evidence of nuclear-plasma interactions.
\end{rSubsection}
% {\bf  Institute for Laser Engineering, Osaka University} \hfill{ Osaka, Japan}
% 
% \vspace*{-.05in}
% {\em Visiting Researcher} \hfill {\bf February -- March,  2015}
% 
% \begin{list2}
% \item    Research and evaluation of solid debris collection diagnostics in search of evidence of nuclear-plasma interactions.
% \end{list2}





\begin{rSubsection}{University of Utah}{\DTMdisplaydate{2010}{8}{11} -- \DTMdisplaydate{2011}{8}{23}}{Undergraduate Researcher, Nuclear Engineering}{Salt Lake City, Utah}
\item Developed simulation of Neutron Activation Analysis, an analytical technique using neutron irradiation of matter to determine highly precise compositions of samples.
\iffull{ \item Simulation optimizes irradiation times of samples to minimize resulting radioactivity.}  \else \fi
\iffull{ \item Presented paper at 2011 ANS Student Conference, 2011 2\textsuperscript{nd} Utah Detection Conference.}  \else \fi
\end{rSubsection}\vspace{-1.5\baselineskip}
\begin{rSubsection}{}{\DTMdisplaydate{2009}{8}{29} -- \DTMdisplaydate{2010}{5}{15}}{Undergraduate Researcher, Chemistry}{}
\item  Synthesis and characterization of metal-doped Cadmium-Selenium quantum dots used to produce photonic crystals structured after iridescent scales of several Brazilian beetles.
\item  Applications include fully-optical circuitry and tunable, customizable photoluminescent sensors for desired molecules and/or cells.
\iffull{ \item  Later research involved sol-gel dip-coating quantum dots for use in geothermal wells.}  \else \fi
\end{rSubsection}

% {\bf University of Utah} \hfill{ Salt Lake City, Utah}
% 
% \vspace*{-.05in}
% {\em Undergraduate Researcher, Nuclear Engineering} \hfill {\bf August, 2010 -- August, 2011}
% 
% \begin{list2}
% \item Developed simulation of Neutron Activation Analysis, an analytical technique using neutron irradiation of matter to determine highly precise compositions of samples.
% \item Simulation optimizes irradiation times of samples to minimize resulting radioactivity.
% \item Presented paper at 2011 ANS Student Conference, 2011 2\textsuperscript{nd} Utah Detection Conference.
% \end{list2}
% 
% 
% {\em Undergraduate Researcher, Chemistry} \hfill {\bf August, 2009 -- May, 2010}
% 
% \begin{list2}
% \item  Synthesis and characterization of metal-doped Cadmium-Selenium quantum dots used to produce photonic crystals structured after iridescent scales of several Brazilian beetles.
% \item  Applications include fully-optical circuitry and tunable, customizable photoluminescent sensors for desired molecules and/or cells.
% \item  Later research involved sol-gel dip-coating quantum dots for use in geothermal wells.
% \end{list2}

% {\bf University of West Florida} \hfill{ Pensacola, Florida}
% 
% \vspace*{-.05in}
% {\em  Visiting Researcher, Department of Physics} \hfill {\bf May, 2008 -- January, 2009}
% % {\em  High School Researcher, Department of Physics} \hfill {\bf May, 2008 -- January, 2009}
% 
% \begin{list2}
% % \item  Investigated and modeled specific heat capacity anomalies of liquid crystals, namely, the 4'-octyl-4-biphenyl-carbonitrile molecule, due to the effect of mesophase transitions.
% \item Modeled specific heat capacity anomalies of 4'-octyl-4-biphenyl-carbonitrile liquid crystals, due to the effect of mesophase transitions.
% \item  Research proceeded to place third in the 2009 Florida State Science Fair, and as a finalist in the 2009 Intel International Science and Engineering Fair.
% \end{list2}


\begin{rSubsection}{University of West Florida}{\DTMdisplaydate{2008}{5}{5} -- \DTMdisplaydate{2009}{1}{18}}{Visiting Researcher, Department of Physics}{Pensacola, Florida}
% \item  Investigated and modeled specific heat capacity anomalies of liquid crystals, namely, the 4'-octyl-4-biphenyl-carbonitrile molecule, due to the effect of mesophase transitions.
\item Modeled specific heat capacity anomalies of 4'-octyl-4-biphenyl-carbonitrile liquid crystals, due to the effect of mesophase transitions.
\iffull{ \item  Research proceeded to place third in the 2009 Florida State Science Fair, and as a finalist in the 2009 Intel International Science and Engineering Fair.}  \else \fi
\end{rSubsection}
